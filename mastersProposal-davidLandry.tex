\documentclass[10pt,letterpaper,oneside]{article}

\input{./latexGoodPractices/preamble}

%==============================================================
% FILL THIS SECTION

\newcommand{\projectTitle}{Assessing the behavior of the Iterative Closest Point algorithm in complex environments}
\newcommand{\projectStudent}{David Landry}
\newcommand{\projectStudentNIP}{111 040 494}
\newcommand{\projectDate}{March 23, 2017} % or manually: November 23, 2017

\newcommand{\projectSupervisor}{Prof. P. Gigu\`{e}re}
\newcommand{\projectCoSupervisor}{Prof. F. Pomerleau}

% Change to your specific bibliography file
%\addbibresource{./latexGoodPractices/exampleReferences.bib}
\addbibresource{./references.bib}
%==============================================================




% ---------------------------------------------------------------
% Load style
\input{mastersProposalStyle.tex}
% ---------------------------------------------------------------

% Colored text
\usepackage[dvipsnames]{xcolor}

% Fill the template with text
\usepackage{lipsum}
\newcommand{\lightlipsum}[1][0]{\textcolor{gray!50}{\lipsum[#1]}}

% Customize enumerate list
\usepackage{enumerate}

% Acronyms
\acrodef{ICP}[ICP]{Iterative Closest Point}
\acrodef{CELLO}[CELLO]{Covariance Estimation and Learning through Likelihood Optimization}
\acrodef{CELLOEM}[CELLO-EM]{Covariance Estimation and Learning through Likelihood Optimization and Expectation Maximizati8,902
Postgraduates	8,689on}
\acrodef{EM}[EM]{expectation maximization}

%================================================================
\begin{document}
\makeCustomTitle
\thispagestyle{titlePage}

% ---------------------------------------------------------------
\section{Introduction}

When thinking about robots of today, one might imagine a robotic arm in a car factory.
This kind of fixed, specialized robots is successfully being used to improve production efficiency in different domains.
However a new type of robot is emerging, one that is free to move in its environment.
The RHINO robot is a good example \cite{Burgard1998}.
These are not easy to design, as they rely on precise and robust localization.
A robot needs to know where it stands with respect to the environment if we want it to act upon its surroundings in a meaningful way.
Although many efforts have been made in that direction in the past, localizing a robot in a given environment is still a challenging research problem.

Various sensing modalities are used for localization.
Two main examples are cameras and lidar sensors.
Cameras capture information about the existing light in the environment and try to localize against it.
They are passive sensors in that sense.
Lidar sensors, however, are active.
They emit light in the environment and capture the emitted signal as it comes back to the sensor.
While cameras are good at detecting and recognizing landmarks, lidars are less sensitive to changes in lighting conditions \cite{McManus2012}.

The need for accurate localization against a map is outlined in the Teach-and-Repeat task.
In this task there is a preliminary \emph{Teach} phase.
During that time the robot registers sensory inputs in a database as it is driven through a trajectory.
In a subsequent \emph{Repeat} phase, the robot repeats the original trajectory by comparing the sensory inputs with those it recorded in the database.
In that sense, the database can be taught of as a map that the robot localizes against later.
Note that this paradigm is sensor-agnostic as it has been successfully implemented using cameras \cite{Furgale2010} and lidar sensors \cite{Sprunk2013}. 

This proposal focuses on lidar sensing as a mean of robust localization.
First, it outlines the problematic of \emph{uncertainty estimation} with lidar sensors in \autoref{sec:problem}.
The importance of this research problem is emphasized in \autoref{ssec:motivation}.
A summary of the existing approaches to solve it is in \autoref{sec:stateoftheart}.
Then, \autoref{sec:objectives} describes my objectives with respect to uncertainty estimation of lidar sensing.
It also contains a reasonable set of steps to accomplish these objectives.

% ---------------------------------------------------------------
\section{Research Problem} \label{sec:problem}

A viable way to localize a robot using the range data from a lidar sensor is the \ac{ICP} algorithm \cite{Zhang1994}.
It was developed simultaneously by \citet{Chen1991} and \citet{Besl1992}.
It consists in iteratively aligning two point clouds by minimizing the distance between the closest points in each of them.
In a first step, each point from the first point cloud is paired with its closest neighbor in the other point cloud.
In a second step, a point cloud is moved such that the error between the pairs of points is minimized.
Those steps are repeated until convergence.
The output is a rigid transformation between the reference frame of both scans.

When minimizing the error between the pairs of points, the error function could simply be the distance between the associated points.
This variant is called the point-to-point version of ICP \cite[section 2.6.2]{Pomerleau2015}.
There is also a point-to-plane version.
In that case, the error is the distance between the point and the plane estimated from the shape of the other point cloud around its closest neighbor.
In both cases, the error minimization is a non-convex combinatorial problem, because it is based on the pairing of closest points.

In the context of mobile robotics, \ac{ICP} can be used to compute odometry or localization \cite{Censi2007}.
It is used as odometry when successive scans are matched to estimate the egomotion of a robot.
It is used as a localizer when a scan is matched with a larger point cloud to identify where the robot is within the larger point cloud.

\subsection{The \acl{ICP} algorithm uncertainty} 

One problem that arises when using \ac{ICP} is the estimation of its uncertainty.
By uncertainty, we mean what kind of error is this algorithm likely to commit with respect to the true transform between the two point clouds.
This will be the main theme of my proposal, which can be framed as follows:

\begin{center}
\emph{What is the uncertainty on the result of the \ac{ICP} algorithm on two given point clouds?}
\end{center}

The answer to this question is complex.
The behavior of \ac{ICP} is greatly conditioned by the geometry of the matched scans \cite{Censi2007}.
The result of this algorithm is also affected by sensor noise and data association problems.
It is common to model this uncertainty as gaussian \cite{Censi2007,Bonnabel2016}.
This assumption is useful to make the computation and representation of the uncertainty tractable.
It also makes it easier to integrate \ac{ICP} in a Kalman Filter \cite{Zhang1994}.
Consequently, the problem can be posed more precisely as the problem of searching for a covariance matrix that would represent the uncertainty.

\subsection{Motivation}
\label{ssec:motivation}

A better understanding of the uncertainty of \ac{ICP} would benefit many areas of research.
As already mentioned, predicting the covariance matrix of \ac{ICP} matches would allow us to integrate them in a Kalman Filter \cite{Zhang1994}.
We would in turn be able to improve our state estimation of the robot.

An improved grasp on the uncertainty of \ac{ICP} would also allow the construction of better maps for Teach-and-Repeat.
In particular, it would enable us to be more careful in areas that are more difficult for \ac{ICP}.
My previous paper \cite{Landry2016} is a good step in that direction.
It can be considered as a map curating algorithm.
It changes the level of detail in a Teach-and-Repeat map according to the uncertainty a robot has when localizing against it.
The level of detail is increased in sections that have high uncertainty, and decreased in sections where the level of precision is satisfactory.
However, it remains difficult to detect the areas that are uncertain for \ac{ICP} in the absence of a reliable and quick algorithm to estimate the covariance matrix.

% ---------------------------------------------------------------
\section{Related Work}

\label{sec:stateoftheart}

As of today, a hand-tuned estimate of the covariance matrix (i.e., a covariance that is independent of any sensor input) is still a common solution to this problem \cite{VegaBrown2013em}.
Finding a good covariance matrix requires a lot of experience and trial and error.
It is also task-specific.
The chosen matrix has to be large enough to accommodate the most difficult match encountered during a task.
However, it also has to be small enough to estimate the state of the robot to a satisfactory degree.

The covariance matrix can also be estimated through sampling.
This is referred to as the \enquote{brute-force} method.
Unfortunately, the computational cost of \ac{ICP} makes sampling many scan matches prohibitive.
Thus, estimating the covariance through sampling cannot be done online \cite{Censi2007}.

\citet{Mehra1970}, \citet{Aghili2016} use \ac{ICP} inside a special form of the Kalman Filter (called Adaptative Kalman Filter).
The Adaptative Kalman Filter is a form of Kalman Filter where the covariance matrix of the readings is modified in function of the gap between the predictions and the measurements.
The caveat of their approach is that it estimates the covariance of the \ac{ICP} through purely statistical means.
It is reactive, instead of being predictive.
Their algorithm is thus still sensitive to outliers and needs to reject matches that seem unlikely.

\citet{Censi2007} devised a way to estimate the covariance of \ac{ICP} by analyzing the underlying error function.
There are indications that this estimation is optimistic in real-world situations \cite{VegaBrown2013em}, although more work is needed to prove that observation conclusively.
This method was validated on simulated 2D datasets.
\citet{Bonnabel2016} provided a theoretical proof of this approach, but also emphasized the need for the underlying error function to be smooth.
Consequently the covariance estimation formula from \cite{Censi2007} should not be used with the point-to-point variant of \ac{ICP}. 
Effectively, the error function produced by the point-to-point variant is not smooth.
Also, this algorithm makes the assumption that \ac{ICP} always converges in the attraction region of the true solution \cite{Censi2007}.
However, it is well known that \ac{ICP} may converge in a local minimum before reaching the true transform between both point clouds \cite{Bonnabel2016}.
Finally, \citet{Manoj2015} extended the closed-form approach to 3D.
They did not add any assumptions doing so.
It means that this technique also makes the true convergence assumption.
The indication from \citet{Bonnabel2016} that this technique should only be applied on the point-to-plan version of \ac{ICP} also holds.

Afterwards, \citet{VegaBrown2013} introduced a data-driven approach to the problem called \ac{CELLO}.
They used pre-existing sensory data to predict the covariance of a localization algorithm via machine-learning.
It is more general than the previously presented approaches as it is applicable to various sensory modalities.
This algorithm requires a predictor function that maps the sensory input in a smaller prediction space.
The output of the predictor should capture properties that are important to covariance prediction.
One caveat of this approach is the need for ground truth data to train the covariance prediction algorithm.
It was used for state estimation in a real-world situation using visual sensors in \cite{Peretroukhin2015}.
To the best of our knowledge, it was not validated on 3D lidar data.

Finally, \citet{VegaBrown2013em} extended \ac{CELLO} to the situation where no ground truth is available.
It was done using the \ac{EM} algorithm.
This approach was validated for lidar on real-world 2D dataset.
It was compared with the approach of \citet{Censi2007}.
Their results suggest that the covariance of this approach is too optimistic.
Again, we believe that it has not been tested against 3D lidar data.

A summary of these six methods is shown in \autoref{tab:summary}.
In this table, strong convergence assumption means that \ac{ICP} is assumed to always converge to the global minimum.
Also, note that the fixed estimate is considered to need pre-existing data because of its trial and error nature.
While this pre-existing data does not need to contain precise ground-truth information, it needs to contain enough information about the \ac{ICP} error so that a human can tune the covariance.
Finally, \enquote{predictive} means that the algorithm does not need to encounter a difficult match online before it can increase the covariance to more appropriate value.

\begin{table}[htbp]
\centering
\caption{Summary of the different methods to estimate \ac{ICP}'s covariance.}
\begin{tabu}{ccccccc}
\toprule
Method & Real-time &  \parbox[t]{2cm}{ \centering Strong convergence assumption} &
                          Predictive & \parbox[t]{2cm}{\centering Need for pre-existing data} &
                          \parbox[t]{2cm}{\centering Need for ground-truth} & 3D \\ \midrule
Hand-tuned               & x      &       & x   & x &   & x \\
Brute-force estimate     &        &       &  x  &   &   & x \\
Adaptative Kalman Filter & x      &       &     &   &   & x \\
Closed-form estimate     & x      & x     & x   &   &   & x \\
\ac{CELLO}               & x      &       & x   & x & x &   \\
\ac{CELLOEM}             & x      &       & x   & x &   &   \\ \bottomrule
\end{tabu}
\label{tab:summary}
\end{table}

As can be seen in \autoref{tab:summary}, the closed-form estimate can predict the covariance of 3D \ac{ICP} matches.
However, as mentioned earlier in \autoref{sec:stateoftheart}, there are indications that this closed-form solution produces optimistic results.
A natural reaction to this state of affairs is to extend the \ac{CELLO} algorithm to the 3D case.
To be able to formally compare the 3D version of \ac{CELLO} and the existing closed form estimate, we also need a well-posed method to evaluate covariance estimation algorithms for \ac{ICP}.


% ---------------------------------------------------------------
\section{Objectives and Methodology}
\label{sec:objectives}

In light of this information, two main objectives can be defined for my master's.
The first objective is to \emph{validate the existing estimations of \ac{ICP}'s covariance experimentally}.
In other words, we need a sound framework to evaluate the quality of the provided estimates.
To reach that goal, we propose the following steps:

\begin{enumerate}[{1.}1]
\item Review the existing methods of uncertainty estimation and familiarize myself with them.
  This means reproducing the algorithms of all the common methods for uncertainty estimation.
  It also means understanding the method of validation used when creating these algorithms, when applicable.
  In particular, it is important to understand both the algorithms and the methods of validation used by the closed-form estimate \cite{Censi2007} and \ac{CELLO} \cite{VegaBrown2013}, as I am likely to compare my techniques to these works in the future.
\item Create a validation method for \ac{ICP} covariance prediction that captures the necessary properties.
\item Validate the performance of the existing covariance estimation methods on small, toy examples.
  For example, we could test the validation method on datasets of synthetic point clouds.
  It is easier to understand the behavior of our validation method if the examples used are easy to comprehend and fully controlled.
  The validation method can then be adapted if needed.
\item Finally, the algorithms should be tested in real-world environments.
  To do so, I will need datasets that have ground truth information.
  That way, it will be possible to determine if the compared techniques encompass the reported error.
\end{enumerate} 

The second objective is to \emph{improve our prediction of the \ac{ICP} covariance by using a machine learning based algorithm on a 3D lidar dataset}.
More concretely, it implies to do the following:

\begin{enumerate}[{2.}1]
  % Adaptation to novel concepts.
\item Familiarize myself with the \ac{CELLO} algorithm.
  Fortunately, this adaptation work is also required in step 1.1.
  % Theoretical work.
\item Create (or find) a 3D point cloud predictor that captures the information necessary to the execution of the \ac{CELLO} algorithm.
  Effectively, the reference papers \cite{VegaBrown2013,VegaBrown2013em} provide very little detail about the utilized predictor.
  The overall characteristics of the predictor do not seem to be easily transferable to 3D.
  Thus, finding a predictor that captures enough information to estimate covariance will be an important step of my master's.
  % Evaluation in controlled environment.
\item Collect data that is appropriate to the \ac{CELLO} algorithm.
  Effectively, this algorithm requires ground truth information of the localization error.
  I will have to research for any existing dataset that has this information.
  Two existing candidates are the \emph{Challenging data sets for point cloud registration algorithms}~\cite{Pomerleau2012} and the \emph{KITTI} dataset~\cite{Geiger2013}.
  We could also collect our own dataset for this purpose.
  This would require equipment that will be described in \autoref{ssec:equipment}.
  % Evaluation in real world environment.
\item Use all the tools described above to run the \ac{CELLO} algorithm on a 3D lidar dataset.
  The performance of the algorithm could be compared to that of a fixed covariance estimate, and the approach of the closed-form estimate \cite{Manoj2015}.
  It would also be good to use the covariance estimation in a real context, by integrating it to a Kalman Filter for example.
\end{enumerate}

Finally, we must not forget the redaction of my master's thesis, which will occupy the last few months of my project.



\subsection{Necessary equipment}
\label{ssec:equipment}

Some specialized equipment may be required for this project.
In the eventuality we need to collect our own dataset, we have access to a Clearpath Husky A200 mobile robot.
The Husky is equipped with a Velodyne HDL-32e lidar sensor.
The ground truth may be collected using a theodolite.
This could be borrowed from the \textit{Département de géomatique}, here at Université Laval.

% ---------------------------------------------------------------
\section{Schedule}

A chronology of how I plan to execute these different objectives is found in \autoref{tab:schedule}.
Note that the number of objectives is optimistic.
This is a way to minimize the risk associated with my master's, in case some objectives turn out to be infeasible.

\begin{table}[htbp]
  \centering
  \caption{Schedule containing the objectives to be accomplished during my master's.
  The numbers are references to the objectives listed in \autoref{sec:objectives}.}
  \begin{tabu}{XXXXX}   
  \toprule 
  Spring 2017 &
  Summer 2017 &
  Fall 2017 &
  Winter 2018 &
  Summer 2018 \\
  \midrule 
  Internship at Örebro Universitet &
  1.1

  1.2

  2.1 &
  1.3
  
  1.4

  2.3 &
  2.2

  2.4 &
  Redaction \\
  \bottomrule 
\end{tabu}
  \label{tab:schedule}
\end{table}

% ---------------------------------------------------------------
\section{Conclusion}

This document described my project for the rest of my master's at Université Laval.
It introduced the key problematic of covariance estimation for the \ac{ICP} algorithm.
The relevance of this problematic was demonstrated through the examples of state estimation and map curating.
Several state of the art techniques were introduced, and their respective caveats was discussed.
Finally, this document described my objectives in more details and presented a reasonable sequence of steps to accomplish them.

% ---------------------------------------------------------------
\printbibliography


\end{document}
